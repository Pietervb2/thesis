\chapter{Optimization}\label{chap::optimization}
\todo[inline]{Hier kijken naar optimalisatie manieren met valves}
\todo[inline]{Valve actuator, pagina 54 phd thesis}
\todo[inline]{Valve authority, kijk op pagina 111 van phd thesis}
\todo[inline]{kijken naar simulation results en daar vind je ook de effecte nvan de valve authority}
\todo[inline]{kijken naar verwerken van regelsystemen afleverset}
\todo[inline]{Dynamic pressure control valve}
\todo[inline]{Bayesian optimisation}

Several options might be suitable to achieve both functions. A distinction can be made between passive and active control valves. The simplest passive valve would be a flow control valve that does not change its opening. The opening is found through an optimization. Another passive control would be to make use of the already present valve and optimize its temperature set point. Danfoss also supplies Pressure Indepenent Control Valves (PICV), which can guarantee a constant flow rate regardless of pressure fluctuations in the system while still maintaining the same pressure loss over the valve. Active control can also take place with the flow control valve and the PICV where the position of the valve can be remotely controlled. Creating the possibility of applying real time control.  