\chapter{Problem formulation and research plan}\label{chap::PoA}
As mentioned in Section \ref{chap::intro}, the objective of this thesis is to lower the return temperature of the decentralized solutions. To achieve this, a dynamic simulation model needs to be developed that captures the thermodynamic and hydraulic behavior of these networks to predict the return temperature. The model will be used to determine optimal control policies for the supply temperature and the overflow mechanism through data-driven optimization techniques to reduce the return temperature. This research focuses on the Cooltower as a use case, however, the findings should also apply to other networks. This chapter presents the problem formulation and outlines the research plan, detailing the approach for building the model and performing the control policy. 

\section{Problem Formulation} 
\subsection{Graph}
The overall model structure of the district heating network was decided to be a strongly connected directed graph. The looped and branched nature of district heating networks makes them well suited for a graph, which is commonly used in the literature \cite{sibeijn2025economic, Krug2020,OPPELT2016336,Simonssongraph}. The strong connectedness indicates that no water leaves the system. The graph is defined as $\mathcal{G}=(\mathcal{N}, \mathcal{E})$, with a set of nodes $\mathcal{N}$ that represent the junctions in the network. In our model, the nodes correspond to volumeless junctions. Nodes are connected by the edges $\mathcal{E} \subseteq \mathcal{N} \times \mathcal{N}$. An edge is a pipe, potentially equipped with a heat exchanger, pump, or valve. 

\subsection{Edges}
The dynamics of an edge ($e \in \mathcal{E}$) can be described with the two PDEs, derived in Sections \ref{sec::hydropipes} and \ref{sec::thermopipes}.

\begin{equation}\label{eq::formom}
\partial_x p_e + \rho_w g \hat{z} + f \frac{8\rho_w L}{\pi^2 D^5}\left|q_e\right|q_e =0
\end{equation}
\begin{equation}\label{eq::forsimpipePDE}
\rho_w c_w A \frac{\partial T_e}{\partial t} + \dot{m}_e c_w \frac{\partial T_e}{\partial x}=\frac{\left(T_a-T_e\right)}{R'} + w_e
\end{equation}
The hydraulics relation can be  discretized $\partial_x p = \frac{\Delta p}{L}$ which leads to
\begin{equation}\label{eq::thermopde}
    \Delta p_e = f \frac{8\rho_w L}{\pi^2 D^5}\left|q_e\right|q_e + \rho_w g L \hat{z}.
\end{equation}
When a pump is connected to an edge, the pressure loss over the edge becomes
% \begin{equation}
%     \Delta p_e = R_{f}\,\dot{m_e}^2 + \rho_w g L \hat{z} - h_e(r_e) 
% \end{equation}
\begin{equation}
    \Delta p_e = R_{f}\,\dot{m_e}^2 + \rho_w g L \hat{z} - \sum^{\text{n}}_{\text{i=1}} \left( c_i(\omega)  \rho \ \dot{V}^{i}_e \right)
\end{equation}

% h_e(r) = c_er_e$ being the pressure increase applied by the pump, $c_e$ the maximum pumping power capacity, and $r_e \in [0,1]$ the pump frequency rate
with $R_{f}$ a combined constant term representing the frictional resistance of the pipe \cite{sibeijn2025economic}. In case of the edge containing a valve, the extra pressure drop over the edge changes accordingly to its definition from Section \ref{sec::valves}. The second Kirchhoff law states that the net pressure difference over a closed loop must be zero. A set of edges connected in a closed loop is defined as $\mathcal{E}_{\text{loop}} = \left\{e \in \mathcal{E}: e \ \text{in closed loop}\right\}$, formulating the relation as:
\begin{equation}
    \sum_{e  \in \mathcal{E}_{\text{loop}}} \Delta p_e = 0.
\end{equation}
The use of the Kirchhoff laws leads to a system of non-linear equations. Hardy-Cross and the Newton-Raphson methods are often used to solve such systems \cite{KUNTUAROVA}. However, in the case of low flow rate, these methods experience weak convergence. The method developed by Arsene et al. \cite{ARSENE}, which can handle lower flow rates and will therefore be used within the simulation models created in this study.


In the thermodynamic PDE, \ref{eq::thermopde}, the term $w$ [J/(m s)] is added, representing the power transfer per meter of pipe with a heat exchanger. The equation will be solved using the Node Method as it is the most widely applied approach in a dynamic district heating network setting. Although alternative methods show promise for improved accuracy or efficiency, they suffer from various limitations such as computational constraints, implementation complexity, or restrictive operating conditions, while lacking the extensive real-world validation that supports the Node Method \cite{KUNTUAROVA}.

\subsection{Nodes}
The first Kirchhoff law states that the sum of flows entering a node equals the sum of flows exiting the node. This results in the conservation of mass over every node due to the assumed incompressibility of the water. For any node $n \in \mathcal{N}$, the edge sets $\mathcal{E}_{\rightarrow n} = \left\{ e \in \mathcal{E} : e \  \text{enters} \ n\right\}$ and $\mathcal{E}_{n\rightarrow } = \left\{ e \in \mathcal{E} : e \  \text{exits} \ n\right\}$ are respectively the entering and exiting sets of edges. The mass conservation is given by
\begin{equation}
    \sum_{e  \in \mathcal{E}_{\rightarrow n}} q_e(t) = \sum_{e  \in \mathcal{E}_{n \rightarrow}} q_e(t).
\end{equation}
Because frictional heat is neglected and heat capacity is assumed constant, the energy balance in every node can be used to define a mixing rule for the node's temperature based on the incoming water temperatures. For every $n \in \mathcal{N}$ it gives
\begin{equation}
    T_n(t) = \frac{\sum_{e  \in \mathcal{E}_{\rightarrow n}} q_e(t) T_e(t)}{\sum_{e  \in \mathcal{E}_{\rightarrow n}} q_e(t)},
\end{equation}
with the nodal temperature being the temperature of all the exiting edges \cite{Krug2020}. 
\begin{equation}
    T_n(t) = T_e(t), \quad \forall \ n \in \mathcal{N}, e \in \mathcal{E}_{n \rightarrow} 
\end{equation}

\subsection{Heat demand}
The individual user heat demand profile is assumed to be deterministic and will be represented by a summation of sines approximating the user demand profile from \cite{Krug2020}. 

\begin{equation}
Q_{d,n_c}(t)=A_1(n_c) \sin(\frac{2 \pi}{T_1(n_c)} t + \phi_1(n_c)) + A_2(n_c)\sin(\frac{2 \pi}{T_2(n_c)} t + \phi_2(n_c))
\end{equation}
with $Q_{d,n_c}(t)$ [W] as the heat demand for individual user $n_c \in \left\{1,2,...,N_c \right\}$ and $N_c$ as the total number of customers. The array of all the individual user profiles is called $Q_d(t)$, and all entries contain a time shift variable $\tau_{n_c}$, introduced to create a difference in timing between the individual user heat demand profiles.

\begin{equation}
Q_d(k) = 
\left[\begin{array}{c}
Q_{d,1}(k-\tau_1) \\
Q_{d,2}(k-\tau_2) \\
\vdots \\
Q_{d,N_c}(k - \tau_{N_c})
\end{array}\right].
\end{equation}

\subsection{Heat production}
The total heat production can be divided into two components: heat extracted from the ATES and backup heat supplied by a larger district heating network. The relation of the heat pump connected to the ATES in combination with the backup network is defined in the energy balance below.
\begin{equation}
    \rho_w c_w q_e(t) \left(T_s(t) - T_r(t)\right) = \rho_w c_w q_{A}(t)\left(T_{s,A} - T_{r,A}\right) + Q_e(t) + Q_{b}(t),
\end{equation}
with the flow rate on the ATES side $q_A(t)$ [m$^3$/s], the supply and return temperature on the ATES side respectively $T_{s,A}$ and $T_{r,A}$ [K] and the electric power delivered to the compressor $Q_e(t)$ [W]. The temperatures $T_{r,A}$ and $T_{s,A}$ are assumed constant. In case the ATES cannot provide enough energy, the backup network fills the energy gap. The amount of heat delivered by the backup network is stated as $Q_{b}(t)$ [W].

\subsection{Control problem}
The developed discrete model for the return temperature of the Cooltower's DCS for time instance $k$ is given by

\begin{align}
& T_r(k) = f(x(k), u(k)) \notag \\
& x(k) = 
\left[\begin{array}{c}
T_{s,1}(k) \\
T_{s,2}(k) \\
 \vdots  \\
T_{s,N_c}(k) \\
\dot{m}_1(k)  \\
\dot{m}_2(k) \\
\vdots \\
\dot{m}_{N_e - 1}(k) \\
\dot{m}_{N_e}(k)  \\
\end{array}\right], \quad x(k) \in \mathcal{X}\\
&u(k) =  \left[\begin{array}{cccc}
T_{\text{set}}(k) & q_A(k) & Q_b(k) & Q_e(k) 
\end{array}\right], \quad u(k) \in \mathcal{U} \notag 
\end{align}
With the supply temperature at delivery set $n_c$ denoted by $T_{s,n_c}$ [K], and the mass flow at edge $e \in \left\{1,2,...,N_e \right\}$ denoted by $\dot{m}_{e}$ [kg/s]. The input space $\mathcal{U}$ is limited by the range of the input variables but also by the state space $\mathcal{X}$ as the mass flow and minimum water temperature are limited based on the time required for the delivery sets to reach a desired temperature. 

This research employs a Bayesian Optimization approach. One of the main reasons for this choice is that standard white-box optimization methods become too computationally intensive when applied to this model. This computational burden stems from the nonlinearity of both the Node Method and the hydraulic system equations. Since the Node Method is non-differentiable and contains integer variables, standard optimization approaches must treat the complete system as a Mixed Integer Non-Linear Problem (MINLP), which is time-consuming to solve [19]. Rather than compromising model accuracy by adopting less computationally intensive thermodynamic methods or simplifying the nonlinear hydraulic equations, Bayesian Optimization circumvents these computational challenges by treating the model as a black box, allowing us to maintain the simulation model's accuracy while enabling efficient optimization. The resulting optimization formulation is stated by 

\begin{align}
\min_{u(k)} & \quad f(u(k)) \\
\text{s.t.} & \quad g_i(u(k)) \leq 0 \quad \forall i =1 ,...,w. \notag \\
\notag
\end{align}
\section{Research Plan}
This section outlines the research approach, discussing the use of simulation software, detailing the step-by-step development of the model, and the implementation of the optimization policy.

\subsection{Simulation software}\label{sec::SimulationModel}
An overview of possibly suitable simulation tools was presented in Table \ref{tab::simsoft}. The paid license tools IDA ICE and TRNSYS could have been of high interest due to their dynamic modeling capability. However, it was decided to proceed with a non-commercial program due to the high cost and the drawback that these programs are not used at Eneco, making implementation more difficult. IDA ICE offers a free educational license, but it will no longer be valid once this thesis is completed. The second argument also holds for the programs EnergyPLAN and EnergyPlus. Besides the fact that EnergyPLAN uses an hourly timestep, and EnergyPlus focuses on building thermal behavior without detailed network modeling. The most serious option of the non-Python tools was OpenModelica, which is an open-source software tool that supports the Modelica language \cite{Modelica}. Although the Modelica language is commonly used for DHN modeling \cite{KUNTUAROVA}, the OpenModelica documentation is not elaborate, making debugging more difficult. Since using it would also require Eneco employees to learn a new modeling language, we decided not to proceed with OpenModelica.

Therefore, the focus was pointed towards Python-based open-source software. PyDHN and GridPenguin support dynamic simulation of the thermodynamics. Nevertheless, PyDHN is still a beta version, and GridPenguin's fixed timestep is 1 hour, making both tools unsuitable. Pandapipes offers a quasi-static approach without a fixed minimum timestep. While it is an interesting option, the time required to understand its coding structure and the potential time debugging would take was considered greater than developing our model from scratch

Therefore, it was determined to make the model from scratch in Python using the Python open-source projects as inspiration for the model structure.

\subsection{Model development}
The construction of the model will be done in a stepwise manner, gradually increasing the level of complexity. Every step will be tested against a couple of base tests checking the conservation of mass in the nodes, the temperature development along the network, and how the pressure and temperature in the system change due to a variation in heat demand. Some steps include additional tests, which are described at each corresponding level. The model development procedure is outlined below

\begin{enumerate}
\item Develop a graph-based framework to represent the network topology.
\item Implement the thermodynamic model for a pipeline.
\begin{itemize}
\item Validate the implementation by comparing it with the dynamic pipe model from the Standard Modelica Library and real pipeline data.
\end{itemize}
\item Construct a basic looped network configuration.
\item Integrate producers, consumers, and heat exchangers into the network.
\item Build a simplified network with two consumers and one producer. \label{networkstep}
\item Incorporate pressure calculations and add components such as valves, pumps, and an overflow mechanism.
\item Repeat step \ref{networkstep} with the updated hydraulic elements.
\item Scale up the network to include 20 consumers and 2 producers.
\begin{itemize}
    \item Verify the model by creating different heat demand profiles through applying time shifts of the peaks. 
\end{itemize}
\item Reproduce the DCS (District Control System) of the Cooltower.
\begin{itemize}
\item Validate internal supply temperatures for each delivery set using available data. Additionally, use supply and demand data, along with return temperature measurements, to assess model accuracy.
\end{itemize}
\end{enumerate}

\subsection{Control Policy}
Once the validated model is obtained, it will be used to generate data to implement Bayesian Optimization. The appropriate acquisition function and covariance kernel must be identified to ensure the accurately representation of the DCS model. The performance of the Bayesian Optimization will then be evaluated across various combinations of control inputs and different model parameters. 

\subsection{Research Questions}
\todo[inline]{dit met Max bespreken, want ik geef al been samenvatting van het problem in de sectie hier na}
The problem formulation and research plan were deduced based on the following research questions.

\begin{itemize}
    \item \textbf{Q1: How accurately can mathematical models of thermal dynamics in small-scale heating networks predict return temperatures, using the Cooltower decentralized system as a case study?}
    \item \textbf{Q2: How can the simulation model of the heating network in the Cooltower be validated?}
    \item \textbf{Q3: How can Bayesian Optimization be applied to lower the return temperature?}
    
\end{itemize}

\subsection{Timeline}
The desired timeline for the project is shown below:

\begin{table}[h]
\center
\begin{tabular}{l|l}
Task           & Week of completion \\ \hline
Model step 1   & Week 39\footnotemark[1]           \\
Model step 2   & Week 39\footnotemark[1]            \\
Model step 3   & Week 40            \\
Model step 4   & Week 40            \\
Model step 5   & Week 41            \\
Model step 6   & Week 42            \\
Model step 7   & Week 43            \\
Model step 8   & Week 44            \\
Model step 9   & Week 46            \\
Control policy & Week 50           \\
First report draft & Week 4 (2026)
\end{tabular}
\end{table}
\footnotetext[1]{During the literature survey, work on these tasks had already been initiated.}
