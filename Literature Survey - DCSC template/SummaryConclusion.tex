\chapter{Summary and conclusions}
This thesis project focuses on the high return temperature from the decentralized solutions
during periods of low heat demand. Decentralized solutions (DCS) are small local heating and cooling networks containing less than 1500 connections. In the case of Eneco, several of these DCS are connected to an Aquifer Thermal Energy Storage (ATES). The heat and cold stored in the ATES need to be respectively in- or decreased by a heat pump to be used in the system. Due to the high return temperature the heat pump efficiency drops and it can even lead to malfunctioning of the heat pump. Eneco wants to have a dynamical model of these networks to obtain insights in their inner working and to see the effect of supply temperature control and/or bypass control in order to lower the return temperature. The DCS of the Cooltower in Rotterdam was chosen as use-case for this project, as it is their project with the most active sensors, providing the data for eventual validation of the model. 

It was decided to make the model from scratch in Python, as it deemed more time efficient and easier to use for other Eneco employees to make a model oneself than use an already existing one. Due to the branched structure of the network, a graph modeling approach will be used. The thermodynamic behavior of the pipes will be simulated using the Node Method, due to its shown accuracy in other research. Where the heat exchanger can be modeled using the effectiveness-NTU method. The hydraulics in the system can be described by two Kirchoff laws. Resulting in a system of nonlinear equations, which will be iteratively solved. A stepwise approach to building the model is proposed.

The dynamic simulation model will be used to perform an optimization over the supply temperature and the bypass control valve to lower the return temperature. Due to the potential high computational intensity of the model it was decided to apply Bayesian optimization. In this way the simulation model could still maintain its accuracy and it could be used for the optimization. The suited covariance kernel and acquisition function needed to be determined during the research. 

