\chapter{System Modeling}\label{chap::sysmodel}
This section outlines the approaches to modeling the thermodynamic and hydraulic behavior of the Decentralized Solution in the Cooltower. It begins with an overview of the overall structure of the model. Subsequently, the hydraulic and thermodynamic modeling of the individual components is discussed. With particular attention given to the modeling of the valves. As unlike the other elements, the overflow mechanism (which is a valve) offers more flexibility within the optimization process, allowing exploration of alternative designs for further research. Afterwards, the peak and demand of the system are analyzed. Finally, a review of existing software tools for simulating district heating networks is presented. 

\section{Preliminaries}\label{sec::preliminaries}
Several general assumptions commonly used in the literature on  district heating networks and some dimesionless variables are outlined below. These assumptions are used throughout this chapter. Additional assumptions specific for particular methods are discussed in their dedicated sections.

This chapter focuses on modeling district heating networks. However, the same approach is applicable to district cooling systems, with modifications to account for the switch from heating to cooling.

\subsection{Modeling assumptions}
One of the most important modeling assumptions is that the hydraulic and thermodynamic systems can be assumed to be decoupled. This can be assumed as the speed of sound in the water is (1481 m/s at 20 $^{\circ}\text{C}$ \cite{speedofsound}) causing the changes in flow rate and pressure to be within seconds in a small district heating network like a DCS. In contrast with the spread of thermodynamic changes as these travel at the speed of the maximum flow rate, which is around 3 m/s. In combination with the assumptions that the water is incompressible, has a constant density and heat capacity, and that frictional heat is negligible, the two systems can be decoupled. 

Another important assumption based on the high speed of sound in water is that the hydraulics are assumed to be in steady-state. 

Furthermore, these are the other assumptions applied in this chapter. 
\begin{itemize}
    \item The fluid flow is turbulent and fully developed.
    \item The ambient temperature is constant along the length of the pipe.
    \item Homogeneous mixing at the pipe junctions.
    \item The pipes are cylindrical and are fully filled with water.
    \item No water leaves the pipe system.
\end{itemize}

% It is assumed that the hydraulics are in a steady state due to the high speed of sound in water (1481 m/s at 20 $^{\circ}\text{C}$ \cite{speedofsound}) causing the changes in flow rate and pressure to be within seconds in a small district heating network like a DCS. Whereas, the thermodynamic changes spread at a much lower speed with a maximum flow rate of around 3 m/s. Furthermore, it is assumed that the water is incompressible, has a constant density and heat capacity, and that frictional heat is negligible. Based on these assumptions, the hydraulic and thermodynamic systems can be assumed to be decoupled. In addition, it is assumed that the pipelines are cylindrical and completely filled with water. 
\todo{citaties naar waar de aannames ook gebruikt worden?}

% Andere aannames die mogelijk hier hier nog kunnen worden benoemd afhankelijk van hoe het gaat met de methods voor de pijpleiding:
% - zelfde diameter van de pijpleiding?
% - constant heat transmission coefficient
% - ambient temperature constant along the length of the pipeline
% - spatially homogenous velocity and temperature in the cross-seciton
% - heat diffusion in axial direction is neglected
% deze hierboven komen allemaal van Maurer.
% - turbulente flow? [Yvo Putter] -> Dit goed checken, moet ook turbulent flow aannemen voor de darcy weisbach
% - fully developed? [Yvo]
% - homogeneous mixing at pipe junctions
% - one directional flow?
\subsection{Dimensionless Numbers}
\textbf{The Reynolds Number (Re)} is the primary parameter correlating the viscous behavior of all Newtonian fluids \cite{white2011fluid}. It is a ratio between the internal and viscous forces within a fluid. And it is used as an indicator for a flow to be laminar or turbulent. 

\begin{equation}\label{eq::Re}
R e=\frac{\rho D v}{\mu}=\frac{4 \dot{m}}{\pi D \mu}
\end{equation}

Where $\rho$ [kg / m$^3$] is the density of the fluid, $v$ [m/s] the fluid velocity and $D$ [m] the diameter of the pipe, $\mu$ its dynamic viscosity [kg / (m $\cdot$ s] and $\dot{m}$ [kg/s] is the mass flow. 

\textbf{The Darcy friction factor (f)} can be found in the Darcy-Weisbach equation, which calculates the pressure flow drop through a pipe using the pipe flow. The coefficient can be determined using the Moody diagram, where $f$ is plotted against the Reynolds number for different relative roughnesses $\epsilon / d$. Or it can be determined analytically using the Colebrook-White equation \ref{eq:CoolbrookWhite}, which is the most popular method, for the transition region (2300 $\leq$ Re $\leq$ 4000) and the turbulent region (Re $\geq$ 4000) in smooth and rough pipes \cite{Darcyfrictionfactor}. This requires an iterative approach to determine $f$. There are also non-iterative methods, but they have a relative error to the Colebrook-White equation, however the authors of \cite{Darcyfrictionfactor} found that for certain correlations the error is negligible.

\begin{equation}\label{eq:CoolbrookWhite}
\frac{1}{\sqrt{f}}=-2 \log \left(\frac{\epsilon}{3.7 D}+\frac{2.51}{\operatorname{Re} \sqrt{f}}\right)
\end{equation}

With the roughness of the pipe $\epsilon$ [m].

\section{Model structure}
In \cite{GUELPA2016586}\cite{KECEBAS2012339}, a black box approach is applied, where the behavior of the entire DHN is put into one function, which is determined using data-driven methods. Another approach is to look at physical laws and parameters to create a model, the white box approach, which is the more conventional option used in the majority of the DHN modeling articles. The grey box model combines these two and can be found in \cite{grey1}\cite{grey2}. This literature survey focuses on the white box model.

District heating networks are branched or looped based networks, making them perfectly suited for modeling with a graph approach. The graph consists of nodes, which are junctions in the system, connected by edges. An edge is a pipe, potentially equipped with a heat exchanger, pump, or valve \cite{sibeijn2025economic}. On the edges, producers can add heat, and consumers can subtract it. The temperature at each node is determined by a weighted average of the incoming mass flows and their respective temperatures. All outgoing edges from the node are assigned the same temperature as the node itself. To calculate the mass flows and pressure within the network, Kirchhoff's two laws must be upheld. The first law asserts that the total mass flow entering and leaving a node must be equal, ensuring mass conservation. The second law requires that the net pressure drop around any closed loop in the network must be zero, maintaining pressure balance. The pressure loss within the pipes depends quadratically on the flow, making the hydraulics a system of non-linear equations. As we assume a decoupled system, the hydraulic problem is computed first, after which the obtained mass flow rates are used for the thermodynamic system. 

To solve this hydraulic system of non-linear equations, we assume that the demand and heat supply are known, leaving the pipe flows to be determined. This can be done by making use of iterative methods like Hardy-Cross \cite{HardyCross} and Newton-Raphson \cite{NewtonenHard}. These methods make an initial guess of the flow and then iteratively adjust this guess until it converges. Where the Hardy-Cross method runs through each loop independently, the Newton–Raphson method runs through all loops simultaneously \cite{NewtonenHard}. In \cite{STEVANOVIC}, the authors claim that they developed a method of square roots for solving the linearized system that outperforms the Hardy-Cross method in convergence time and validated it with real data. Also the aggregated models from \cite{LARSEN2002995} (also tested with real data) converge quicker (which is mainly interesting for big district heating networks) and overcome some of the limitations of the Hardy-Cross method.

The thermodynamic behavior of the edge in the system is described by a PDE. Different solutions for this equation are discussed in more detail in Section \ref{sec::thermo}

\section{Hydraulic modeling}
It is essential to know the pressure loss across all components, as this information, combined with Kirchhoff's laws, enables the determination of fluid flow throughout the district heating network. 

The total pressure loss of a pipe system can be divided into major and minor losses. The major pressure loss ($\Delta p_p$) is caused by friction along the length of the pipe and the height difference, the minor losses ($\Delta p_m$) are the result of the entrance or exit of the pipe, fittings, bends, valves, and sudden expansions and contractions of the pipes. Although they are called minor losses, a partially closed valve can cause a greater head loss than a long pipe \cite{white2011fluid}. Adding the heat exchangers ($\Delta p_{hex}$) to the major losses in this pipe system results in the following definition of total pressure change within a loop of the system. 

\begin{equation}
    \Delta p_{tot} = \sum \Delta p_{p} + \Delta p_m + \sum \Delta p_{hex} 
\end{equation}
The pressure loss over the pipes is discussed in Section \ref{sec::pipes}. Where $p_m$ depends on the type and size of the heat exchanger \cite{YvoPutter}, which needs to be determined for the DCS of the Cooltower. The minor losses are defined according to Equation \ref{eq::minorpres}.

\begin{equation}\label{eq::minorpres}
    \Delta p_m = \frac{\rho V^2}{2g} \sum \zeta
\end{equation}

The fluid velocity $V$ [m/s] and the loss coefficient $\zeta$ [-]. This formula holds that when all pipes have the same diameter, if the diameter changes, the fluid velocity will be affected. In that case, the losses need to be added separately. The manufacturer often provides the loss coefficients. However, the term is not based on the Reynolds Number or the roughness of the pipe, but solely on its diameter and assuming a turbulent flow \cite{white2011fluid}. If adding minor losses for a large pipe system becomes too cumbersome, one might choose to apply an additional percentage (5 to 20\%) on pipe friction losses \cite{echtephdthesis}. The pressure loss of the valves is discussed in more detail in Section \ref{sec::valves}.

%% ## HEAD
% In the literature, head (Equation \ref{eq::head}) is often used instead of pressure to describe hydraulic losses. 
% \begin{equation}\label{eq::head}
%     h = \frac{p}{\rho g}
% \end{equation}
% With head $h$ [m], pressure $p$ [Pa], gravitational constant $g$ [m/s$^2$], and the height to a certain reference point $z$ [m]. 

% The total head loss of a pipe system can be divided into major and minor losses. The major loss ($h_f$) is caused by the friction along the length of the pipe, the minor losses ($h_m$) are the result of the pipe entrance or exit, fittings, bends, expansions and contractions of the pipes and valves. Although they are called minor losses, a partially closed valve can cause a greater head loss than a long pipe \cite{white2011fluid}. Adding the heat exchangers ($h_{hex}$ to the major losses in this pipe system results in the following definition of the total head. 
% \begin{equation}
%     \Delta h_{tot} = h_f + \sum h_m + h_{hex} 
% \end{equation}
% The friction head loss $h_f$ is discussed in Section \ref{sec::pipes}. Where $h_m$ depends on the type and size of the heat exchanger \cite{YvoPutter}, which needs to be determined for the DCS of the Cooltower. The minor losses are defined according to Equation \ref{eq::minorhead}.

% \begin{equation}\label{eq::minorhead}
%     \sum h_m = \frac{V^2}{2g} \sum K
% \end{equation}

% The fluid velocity $V$ [m/s] and the loss coefficient $K$ [-]. This formula holds when all the pipes have the same diameter, as if the diameter changes, it affects the fluid velocity. In that case the losses need to be added separately. The loss coefficient is often provided by the manufacturer. However, the term is not based on the Reynolds Number or the roughness of the pipe, but solely on its diameter and assuming a turbulent flow \cite{white2011fluid}. The head loss of the valves is discussed in more detail in Section \ref{sec::valves} to further discover the modeling possibilities for the overflow mechanism. 


\subsection{Pipes}\label{sec::pipes}
An approximation of the 1-dimensional compressible Euler equations for cylindrical pipes is used to describe the dynamics within a pipe \cite{Krug2020}. The first two equations, the continuity equation and the 1D momentum equation, are needed for the calculation of the mass flows. They are stated down below.

\begin{equation}
\partial_t \rho+\partial_x\left(\rho v\right)=0
\end{equation}
\begin{equation}
\partial_t (\rho v)+\partial_x(\rho v^2)+ \partial_x p+\rho g \hat{z}  +f \frac{\rho}{2 D}|v| v=0
\end{equation}
Where $\rho$ [kg/m$^3$] is the water density, the water velocity $v$ [m/s], the pressure in the pipe $p$ [Pa], $g$ the gravitational acceleration [m/s$^2$], the slope of the pipe $\hat{z}$ [-], friction coefficient $f$ [-] and the diameter of the pipe $D$ [m]. According to the assumptions stated in Section \ref{sec::preliminaries} the following holds: $\partial_x v = 0, \partial_t \rho = 0$, $\partial_x \rho = 0$ and $\partial_t v = 0$. Making the continuity equation obsolete and results in the approximation of the momentum equation below \cite{sibeijn2025economic}. 
\begin{equation}\label{eq::mom}
\partial_x p + \rho g \hat{z} +f \frac{\rho}{2 D}\left|v\right| v=0
\end{equation}
Substituting $v = \frac{4 q}{\pi D^2}$ into \eqref{eq::mom} and discretizing $\partial_x p = \frac{\Delta p}{L}$ gives:
\begin{equation}
    \Delta p = p_{p} =  f \frac{8\rho L}{\pi^2 D^5}\left|\dot{m}\right| \dot{m} + \rho g L \hat{z}
\end{equation}
With the mass flow $\dot{m}$ [kg/m$^3$] and the length of the pipe $L$ [m]. The first term on the right side of the equation is equal to the Darcy-Weisbach equation, which describes the pressure loss due to the friction of the fluid with the pipe wall.  

%% ###HEAD
% Rewriting the equation into the head loss gives:

% \begin{equation}
%     h_f = f \frac{8 L}{\pi^2 d^5 g}\left|q\right| q + \hat{z}
% \end{equation}


\subsection{Valves}\label{sec::valves}
Valves regulate the pressure or flow in a pipe system. The Cooltower's heat distribution system contains multiple different types of valves. The pressure losses due to the valves can be calculated in a slightly different manner than the general relation for minor pressure loss in Equation \ref{eq::minorpres}, stated in Equation \ref{eq::realminorpres}, giving a more precise definition for the valves \cite{Artikelphdchris}. 

\begin{equation}\label{eq::realminorpres}
    \Delta p_{v} = \left( \frac{\dot{V}}{K_v} \right)^{2} \frac{\rho}{\rho_{ref}}    
\end{equation}
With the hydraulic conductivity parameter $K_v$  [m$^3$/h/$\sqrt{bar}$], the liquid flow $\dot{V}$ [m$^3$/h], the actual fluid density during the simulations $\rho$, and the fluid density during the manufacturer's $K_v$ measurements $\rho_{ref}$ (typically 1000 kg/m$^3$). The calculation of the valve characteristic $K_v$, which varies with the valve's opening position, depends on the valve type. It is important to mention that for small valve lifts, the $K_v$ value cannot be determined accurately because it exceeds the required tolerances of the basic shape of the valve. The valve's control range is determined up to a 10 percent deviation of the basic shape. When it goes outside of this control range, the valve becomes unpredictable. The manufacturer provides the control range \cite{echtephdthesis}.

\subsubsection{2-Way Control Valves}
At first, we have the 2-way control valves focused on controlling the flow. When this valve is a basic linear 2-way valve, $K_v$ can be determined using Equation \ref{eq::linear2valve}. Equation \ref{eq::equalpercentvalve} applies to an equal percentage valve. Both equations are in accordance with the VDI/VDE2173 standard \cite{VDI/VDE2713}. 

\begin{equation}\label{eq::linear2valve}
    \frac{K_v}{K_{vs}} = \left( 1 -\frac{K_{v,0}}{K_{vs}} \right) \frac{h_{lift}}{h_{\infty}} + \frac{K_{v,0}}{K_{vs}}
\end{equation}
\begin{equation}\label{eq::equalpercentvalve}
    \frac{K_v}{K_{vs}} = {\left(\frac{K_{vs}}{K_{v,0}} \right)}^{\frac{h_{lift}}{h_{\infty}} - 1}
\end{equation}

With the $K_v$ value for the completely opened value $K_{vs}$, the $K_v$ value at the points where the basic shape of the valve characteristic intersects with the y-axis $K_{v,0}$ and the dimensionless valve displacement $\frac{h_{lift}}{h_{\infty}}$ which ranges from 0 to 1 \cite{Artikelphdchris}. The differential pressure control valve is a specialized type of control valve designed to maintain a predefined pressure difference across a pipe. Unlike the previously discussed control valves, it directly controls the pressure drop over the pipe, not its flow. However, in practical applications, the valve is not ideal. The actual pressure difference across it may deviate from the setpoint. The modeling of such non-ideal behavior is further elaborated in \cite{echtephdthesis}.

\subsubsection{3-Way Control Valves}
There are also 3-way valves within the system. For these valves the hydraulic resistance is defined over two valve entrance ports A and B.
The valve characteristic pair can be complementary. In the case, it is a 3-way linear valve port A can be described using Equation \ref{eq::linear2valve} and port B using Equation \ref{eq::3way-valve}. Respectively, changing the valve characteristic formula into Equation \ref{eq::equalpercentvalve} results in a 3-way equal percentage valve. However, sometimes the ports contain two different valve characteristics. 

\begin{equation}\label{eq::3way-valve}
    \frac{K_{v,B}}{K_{vs}} = 1 - \frac{K_{v,A}}{K_{vs}}
\end{equation}

The heat interface units present in the Cooltower, the Arctic WKW-S 4P \cite{fortes_wkw_s_4p}, contain a 6-way ball valve. This valve can be modeled as two interconnected 3-way valves with synchronized control logic. In this simplified model, each valve operates in a binary manner, either fully open or fully closed. Together, they determine whether the heat interface unit is in heating or cooling mode. Sometimes, the 6-way valve can also perform flow control, but not in our case.

\subsubsection{Overflow Mechanism}
\todo[inline]{Kan hier mogelijk nog wat zeggen over de square opening valve}
\todo[inline]{find out what type of valve is really in the system.}

Other more common valves present in hydraulic systems are 

Modulating valve [DHN bijbel p 371]
(three port) control valve -> the control valves 
thermostatic radiator valves
throttling valves

\subsection{Pumps}
Pumps supply the pressure necessary for the fluid to flow through the network. Centrifugal pumps are the most
commonly used pumps in district heating networks. The pump can be described by the pump characteristic, in which the pump pressure or head is plotted against the flow rate. The plot is for multiple rotation pump rotation speeds. The system head curve is also plotted, the system operation points are where both curves intersect. 
\begin{figure}
    \centering
    \includegraphics[width=0.5\linewidth]{}
    \caption{Caption}
    \label{fig:enter-label}
\end{figure}

\section{Thermo-Dynamic modeling}\label{sec::thermo}
\subsubsection{Node Method}
\subsubsection{Lagrangian approach}
\subsubsection{other methods for pipes}
\subsection{Thermal conductivity}
It is assumed that the fluid velocity is high enough to neglect heat conduction in the direction of the flow. Therefore only the heat convection of the fluids needs to be taken into account as well as the heat conduction of the plate material. Uo is the overall heat transfer coefficient between the two fluids. This heat transfer coefficient is the reciprocal of the overall heat transfer resistance (3-13). 
\todo[inline]{van de literatuur studies Femke Jansen}
\subsection{Heat Exchanger}
LMTD, NTU
over nadenken hoe je dit dan doet voor zo'n afleverset
\subsection{Other components}

\section{Head supply}
CHP en ATES

\section{Head demand}


\section{Simulation Software}
- tabel voor alle mogelijke simulatie software die gebruiken van word
- daarin verwijzen welke technologie ze gebruiken
- cfd programmas ook naar gekeken maar te uitgebreid. 

