%
% A Real Chapter
\chapter{System Modeling}\label{chap::sysmodel}
static vs dynamic vs pseudo transient



\section{Thermo-Dynamic modeling}
\subsection{Pipes}
\subsubsection{Node Method}
\subsubsection{Lagrangian approach}
\subsubsection{other methods for pipes}
\subsection{Thermal conductivity}
It is assumed that the fluid velocity is high enough to neglect heat conduction in the direction of the flow. Therefore only the heat convection of the fluids needs to be taken into account as well as the heat conduction of the plate material. Uo is the overall heat transfer coefficient between the two fluids. This heat transfer coefficient is the reciprocal of the overall heat transfer resistance (3-13). 
\todo[inline]{van de literatuur studies Femke Jansen}
\subsection{Heat Exchanger}
\subsection{Other components}
\subsection{ATES}
\section{Hydraulic modelling}
\subsection{Pipes}
\subsection{Heat Exchanger}
\subsection{Other components}
\todo[inline]{Need to include district heating AND cooling}
\section{Overflow mechanisms}
\section{Simulation Software}



% This is real chapter for \ac{DCSC}, ok? We will use it as a demo for the different headings you can use to structure your text.


% \section{First Section}

% This is the section. Referring to equations, figures and tables can easily be done by the commands \verb"\eqnref{}", \verb"\figref{}" and \verb"\tabref{}".

% \begin{equation}\label{eq:First}
% 	H(s) = \frac{1}{s+2}
% \end{equation}

% You see? Refer to equations like this \eqnref{eq:First}.


% \subsection{The first subsection}

% Subsections are the last type of sectioning that is numbered. 


% \subsubsection[Subsection Short Title]{The first sub-subsection with a very very very long title, but in the Table of Contents one can only see the short title}

% Quick! Check the Table of Contents! Nice, ain't it?\index{Nice}


% \paragraph{A paragraph title}

% Subdividing your text in sections and paragraphs automatically makes it nice and structured.